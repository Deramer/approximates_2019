\documentclass[a4paper,12pt]{article}
\usepackage{graphicx}
\usepackage{bm,amssymb}
\usepackage{mathrsfs}
\usepackage[unicode,colorlinks=true,filecolor=blue, menucolor=black, linkcolor=black, citecolor=black,pagebackref=white]{hyperref}
\usepackage[utf8]{inputenc}
\usepackage[russian]{babel}
\usepackage{amsmath}
\usepackage{caption}
\usepackage[left=2cm,right=2cm, top=2cm,bottom=2cm,bindingoffset=0cm]{geometry}
\begin{document}
\title{Семинар по теме: <<Метод стационарной фазы>>}
\maketitle

\subsection*{Задача 1}

Найдём энергию гравитационного взаимодействия шара массы $M$ радиуса
$r$ и точечной частицы массы $m$, которая находится на расстоянии
$R>r$ от него.


\subsubsection*{Решение}

Результат известен из курса общей физики; эта энергия совпадает с
энергией взаимодействия двух точечных частиц: $E=G\frac{Mm}{R}$ ($G$
--- гравитационная постоянная). Получим этот результат непосредственным
вычислением. Из закона приятяжения следует, что энергия выражается
в виде следующего интеграла по шару:
\[
E=G\iiint_{\Omega}\frac{\rho m}{\left|\mathbf{r}-\mathbf{R}\right|}d^{3}\mathbf{r}
\]

 \noindent
где вектор $\mathbf{R}$ направлен на точечную массу, $\rho=\frac{3M}{4\pi r^{3}}$
--- плотность, а область интегрирования $\Omega$ представляет собой
шар. Введем систему координат - поместим шар в начало координат; ось
$Oz$ направим на частицу, так что её координаты $\mathbf{R}=(0;0;R)$.
Запишем этот интеграл в сферических координатах: $d^{3}r=\rho^{2}d\rho\sin\theta d\theta d\varphi$,
причём вектор $\mathbf{r}=\left(\rho\sin\theta\cos\varphi;\rho\sin\theta\sin\varphi;\rho\cos\theta\right)$.
Получаем:

\[
E=G\frac{3Mm}{4\pi r^{3}}\int_{0}^{r}\rho^{2}d\rho\int_{0}^{2\pi}d\varphi\int_{0}^{\pi}\sin\theta d\theta\frac{1}{\sqrt{(R-\rho\cos\theta)^{2}+\rho^{2}\sin^{2}\theta}}
\]

 \noindent
Интеграл по $\varphi$ даёт просто $2\pi$. Раскрывая скобки, видим,
что в знаменателе стоит $R^{2}+r^{2}-2Rr\cos\theta$. Делая замену
в получившемся интеграле $z=-\cos\theta\Rightarrow dz=\sin\theta d\theta$,
мы приходим к следующему интегралу:

\begin{multline*}
E=G\frac{3Mm}{2r^{3}}\int_{0}^{r}\rho^{2}d\rho\int_{-1}^{1}\frac{dz}{\sqrt{R^{2}+\rho^{2}+2R\rho\cdot z}}=\\
=G\frac{3Mm}{2r^{3}}\int_{0}^{r}\rho^{2}d\rho\frac{1}{R\rho}\left.\sqrt{R^{2}+\rho^{2}+2R\rho z}\right|_{z=-1}^{1}=\\
=G\frac{3Mm}{2r^{3}}\int_{0}^{r}\rho^{2}d\rho\frac{1}{R\rho}(R+\rho-|R-\rho|)\underset{R>r\geq\rho}{=}G\frac{3Mm}{r^{3}R}\int_{0}^{r}\rho^{2}d\rho=\frac{GmM}{R}
\end{multline*}



\subsection*{Задача 2}

Найдём момент инерции тора большого радиуса $R$ и малого радиуса
$r$ относительно оси, перпендикулярной ``плоскости тора''.


\subsubsection*{Решение}

Введём координаты на торе:
\[
\begin{cases}
x & =(R+\rho\cos\theta)\cos\varphi\\
y & =(R+\rho\cos\theta)\sin\varphi\\
z & =\rho\sin\theta
\end{cases}
\]

 \noindent
При этом $\varphi\in\left[0;2\pi\right]$, $\theta\in\left[0;2\pi\right]$,
$\rho\in\left[0;r\right]$. Найдём якобиан перехода к таким координатам:
\begin{multline*}
J={\rm det}\left|\begin{array}{ccc}
\frac{\partial x}{\partial\rho} & \frac{\partial y}{\partial\rho} & \frac{\partial z}{\partial\rho}\\
\frac{\partial x}{\partial\varphi} & \frac{\partial y}{\partial\varphi} & \frac{\partial z}{\partial\varphi}\\
\frac{\partial x}{\partial\theta} & \frac{\partial y}{\partial\theta} & \frac{\partial z}{\partial\theta}
\end{array}\right|={\rm det}\left|\begin{array}{ccc}
\cos\theta\cos\varphi & \cos\theta\sin\varphi & \sin\theta\\
-(R+\rho\cos\theta)\sin\varphi & (R+\rho\cos\theta)\cos\varphi & 0\\
-\rho\sin\theta\cos\varphi & -\rho\sin\theta\sin\varphi & \rho\cos\theta
\end{array}\right|=\\
=(R+\rho\cos\theta)\rho(\cos^{2}\theta\cos^{2}\varphi+\sin^{2}\varphi\sin^{2}\theta+\sin^{2}\theta\cos^{2}\varphi+\cos^{2}\theta\sin^{2}\varphi)=\rho(R+\rho\cos\theta)
\end{multline*}

 \noindent
Посчитаем объём тора $\Omega$:
\[
V=\iiint_{\Omega}d^{3}\overline{r}=\int_{0}^{r}\rho d\rho\int_{0}^{2\pi}d\varphi\int_{0}^{2\pi}d\theta\left(R+\rho\cos\theta\right)=\pi r^{2}\times2\pi R
\]

 \noindent
поэтому плотность тора равна $\frac{M}{\pi r^{2}\times2\pi R}$. Момент
инерции тора определяется как:
\begin{multline*}
I_{z}=\frac{M}{\pi r^{2}\times2\pi R}\iiint_{\Omega}(x^{2}+y^{2})d^{3}\mathbf{r}=\frac{M}{\pi r^{2}\times2\pi R}\int_{0}^{r}\rho d\rho\int_{0}^{2\pi}d\varphi\int_{0}^{2\pi}d\theta(R+\rho\cos\theta)^{3}=\\
=\frac{M}{\pi r^{2}\times R}\int_{0}^{r}\rho d\rho\int_{0}^{2\pi}d\theta(R^{3}+3R^{2}\rho\cos\theta+3R\rho^{2}\cos^{2}\theta+\rho^{3}\cos^{3}\theta)
\end{multline*}

 \noindent
Нечётные степени косинуса будучи усреднены по периоду дают ноль. Кроме
того, известно, что $\left\langle \sin^{2}\theta\right\rangle =\left\langle \cos^{2}\theta\right\rangle =\frac{1}{2}$;
поэтому интеграл по $\theta$ легко берётся и остаётся:
\[
I_{z}=\frac{2M}{r^{2}}\int_{0}^{r}\rho d\rho(R^{2}+\frac{3}{2}\rho^{2})=M\left(R^{2}+\frac{3}{4}r^{2}\right)
\]



\subsection*{Задача 3}

Найдём площадь $\sigma_{n}$ единичной сферы в $n$-мерном пространстве.


\subsubsection*{Решение}

Рассмотрим $n$-мерный Гауссов интеграл:
\[
I=\iiint_{\mathbb{R}^{n}}e^{-|\mathbf{x}|^{2}}d^{n}\mathbf{x}
\]

 \noindent
С одной стороны, если рассмотреть его в $n$-мерных декартовых координатах,
мы получим:

\[
I=\int_{-\infty}^{\infty}dx_{1}\dots\int_{-\infty}^{\infty}dx_{n}e^{-x_{1}^{2}-\dots-x_{n}^{2}}=\pi^{n/2}
\]

 \noindent
С другой стороны, мы можем его же рассмотреть в $n$-мерных сферических
координатах. Поскольку подынтегральная функция зависит лишь от расстояния
до центра, то мы можем сразу взять интеграл по всем углам; несложно
заметить, что полученный интеграл даст нам как раз площадь единичной
сферы в $n$-мерном пространстве (например, в трехмерном пространстве
$d^{3}\mathbf{x}=4\pi x^{2}dx$, и $4\pi$ - как раз площадь единичной
сферы в трёхмерье). Таким образом, мы можем записать:

\[
I=\int_{0}^{\infty}e^{-x^{2}}\sigma_{n}x^{n-1}dx=\begin{vmatrix}x=\sqrt{t}\\
dx=\frac{dt}{2\sqrt{t}}
\end{vmatrix}=\frac{\sigma_{n}}{2}\int_{0}^{\infty}t^{\frac{n}{2}-1}e^{-t}dt=\frac{1}{2}\sigma_{n}\Gamma\left(\frac{n}{2}\right)
\]


\[
\sigma_{n}=\frac{2\pi^{n/2}}{\Gamma\left(\frac{n}{2}\right)}=\begin{cases}
\frac{2\pi^{k}}{(k-1)!}, & n=2k\\
\frac{2^{k}\pi^{k-1}}{(2k-3)!!}, & n=2k-1
\end{cases}=\begin{cases}
2, & n=1\\
2\pi, & n=2\\
4\pi, & n=3\\
2\pi^{2}, & n=4
\end{cases}
\]

 \noindent
Мы видим, что известные нам результаты (для $n=1,2,3$) воспроизводятся.


\paragraph{Комментарий про размерную регуляризацию}

Аналитичность полученного результата как функции параметра $n$ позволяет
рассматривать такие, на первый взгляд абсурдные вещи, как ``площадь
сферы в пространстве размерности $n=4-\epsilon$'', при нецелых $\epsilon$.
На этом основан один из способов регуляризации расходящихся интегралов
- так называемая ``размерная регуляризация'', о которой рассказывалось
ранее. 
 \noindent
Допустим, нас интересует интеграл в пространстве размерности $n=4$,
и он расходится. Тогда можно формально доопределить $n$-мерный интеграл
на нецелые $n$, как замену $d^{n}\mathbf{x}$ на $\sigma_{n}x^{n-1}dx$;
и при этом может оказаться, что интеграл сходится при любых $n<4$.
Это позволяет формально рассмотреть интеграл в пространстве размерности
$n=4-\epsilon$ (где интеграл сходится), а затем устремить $\epsilon\to0$.

\subsection*{Контурные интегралы}
Часто в разных физических задачах встречается необходимость интегрировать те или иные функции вдоль кривых. Среди примеров и работа силы при перемещении тела вдоль той или иной траектории, и вычисление массы какой-нибудь проволоки, и циркуляция электрического или магнитного поля в уравнениях Максвелла. Отдельная история про интегралы вдоль различных контуров встречается в теории функций комплексного переменного: там это основной рабочий объект.\\
Выделяют два основных типа контурных интегралов: первого и второго рода.\\\\
\textbf{Контурные интегралы первого рода} - это интегралы от скалярных функций по кривым. Самый простой пример - вычисление их длин. Пусть, например, кривая задана функцией $f(x)$, и мы хотим посчитать ее длину при изменении $x$ от $a$ до $b$. Разобъем кривую на $N$ маленьких кусочков с длинами $\Delta l_{i}$, где $i$ лежит от $1$ до $N$. Тогда выражение для полной длины $L$ дается очевидным равенством
\[
L	=\sum_{i=1}^{N}\Delta l_{i}
\]
Представим $\Delta l_{i}$ по теореме Пифагора как
\[
\Delta l_{i}	=\sqrt{\left(\Delta x_{i}\right)^{2}+\left(\Delta f_{i}\right)^{2}}
\]
где $\Delta f_{i}=f(x_{i}+\Delta x_{i})-f(x_{i})$. Вынесем $\Delta x_{i}$ из под знака корня и получим
$\Delta l_{i}	=\Delta x_{i}\sqrt{1+\left(\frac{\Delta f_{i}}{\Delta x_{i}}\right)^{2}}$
\noindent
Таким образом, выражение для длины кривой в пределе бесконечно мелкого разбиения имеет вид

\[
L	=\int_{a}^{b}dx\sqrt{1+\left(\frac{df}{dx}\right)^{2}}.
\]
Если же мы хотим посчитать какую-нибудь более интересную величину, например, массу кривой, то ответ будет выглядеть иначе. Пусть плотность массы $\rho$, определяемая как коэффициент пропорциональности между массой каждого отдельного кусочка и его длиной

\[
\rho_{i}	=\frac{\Delta m_{i}}{\Delta l_{i}},
\]
каким-то образом меняется от кусочка к кусочку, то есть задается функцией $\rho(x)$. Тогда полная масса кривой имеет вид

\[
M	=\int_{a}^{b}dx\rho(x)\sqrt{1+\left(\frac{df}{dx}\right)^{2}}
\]
\bigskip
\noindent
Однако, иногда кривая не задается какой-то однозначной функцией (например, если она имеет самопересечение). В таком случае, ее нужно запараметрозовать тем или иным способом. Пусть она задана как некоторая векторная функция $\vec{r}(t)$,	где $t\in(t_{A},t_{B})$. Начальную и конечную точки обозначим как $\vec{r}_{A}=\vec{r}(t_{A})$ и $\vec{r}_{B}=\vec{r}(t_{B})$. Длину кривой в этом случае можно вычислить как
\[
L	=\int_{\vec{r}_{A}}^{\vec{r}_{B}}|d\vec{r}|=\int_{t_{A}}^{t_{B}}dt|d\vec{r}/dt|
\]
а, например, ее массу как
\[
M	=\int_{t_{A}}^{t_{B}}dt\rho(\vec{r}(t))|d\vec{r}/dt|
\]
\noindent
\textbf{Контурные интегралы второго рода} в свое время - это интегралы от векторных функций по кривым. Они задаются подобно интегралам первого рода, однако вместо модуля "скорости"  $d\vec{r}/dt$ в них фигурирует скалярное произведение внешнего поля (которое мы интегрируем) на вектор "скорости".  Например, для работы силы $\vec{F}(\vec{r})$ по кривой $\vec{r}(t)$ имеем
$$
A	=\int_{t_{A}}^{t_{B}}dt\left(\vec{F}(\vec{r}(t))\cdot d\vec{r}/dt\right)
$$
\subsection*{Задача 4}
Пусть кольцо из проволоки радиуса $R$ лежит в плоскости $xy$. Представим его кривой $\vec{r}=\vec{r}(\phi)$, где компоненты вектора заданы как
$x	=R\cos\phi$, $y=R\sin\phi$, $\phi\in(0,2\pi)$. Допустим, что это кольцо неравномерно электрически заряжено: линейная плотность заряда задается выражением
$$\rho(\phi)	=\rho_{0}\cos\phi,$$
где $\rho_{0}$ - постоянная, имеющая размерность [заряд/метры]. Необходимо найти потенциал, создаваемых этим колечком в произвольной точке $\vec{r}_{0}$ плоскости $xy$, заданной как
$$x_{0}	=r_{0}\cos\theta,\quad y_{0}=r_{0}\sin\theta$$
при условии $r_{0}\gg R$.

\subsubsection*{Решение}
Давайте запишем выражение, для потенциала, известное из электростатики, как
$$
\Phi(x_{0},y_{0})	=\int|d\vec{r}|\frac{\rho}{|\vec{r}_{0}-\vec{r}|}.
$$
Плотность заряда устроена таким образом, что на одной половине колечка она положительна, а на другой половине - отрицательна, причем суммарный заряд равен 0. Такая конфигурация зарядов называется диполем.\\
Здесь интегрироване ведется по кривой $\vec{r}=\vec{r}(\phi)$, задающей кольцо. Вводя параметризацию углом, описанную в условии, мы переписываем интеграл как
$$\Phi(x_{0},y_{0})	=\int_{-\pi}^{\pi}d\phi|d\vec{r}/d\phi|\frac{\rho_{0}\cos\phi}{|\vec{r}_{0}-\vec{r}|}$$
Непосредственное вычисление дает $|d\vec{r}/d\phi|	=R$,
а
$$
|\vec{r}_{0}-\vec{r}|	=\sqrt{(R\cos\phi-r_{0}\cos\theta)^{2}+(R\sin\phi-r_{0}\sin\theta)^{2}}=\sqrt{R^{2}+r_{0}^{2}-2Rr_{0}(\cos\phi\cos\theta+\sin\phi\sin\theta)}=
$$
$$
	=\sqrt{R^{2}+r_{0}^{2}-2Rr_{0}\cos(\phi-\theta)}.
$$
Тогда интеграл можно переписать как
$$
\Phi(x_{0},y_{0})	=R\rho_{0}\int_{-\pi}^{\pi}d\phi\frac{\cos\phi}{\sqrt{R^{2}+r_{0}^{2}-2Rr_{0}\cos(\phi-\theta)}}.
$$
Сделаем сдвиг переменной $\phi-\theta=\xi$. Тогда получим
$$
\Phi(x_{0},y_{0})	=R\rho_{0}\int_{-\pi}^{\pi}d\xi\frac{\cos(\xi+\theta)}{\sqrt{R^{2}+r_{0}^{2}-2Rr_{0}\cos\xi}}.
$$
Пределы интегрирования не изменились, потому что мы интегрируем периодическую функцию по всему ее периоду. Раскладывая косинус:
$$
\Phi(x_{0},y_{0})	=R\rho_{0}\int_{-\pi}^{\pi}d\xi\frac{\cos\xi\cos\theta-\sin\xi\sin\theta}{\sqrt{R^{2}+r_{0}^{2}-2Rr_{0}\cos\xi}}.
$$
Обратим внимание, что $\theta$ - это просто постоянная, которая определяет то, в какой точке мы смотрим потенциал. Одна из частей интеграла 
$$
-R\rho_{0}\int_{-\pi}^{\pi}d\xi\frac{\sin\xi\sin\theta}{\sqrt{R^{2}+r_{0}^{2}-2Rr_{0}\cos\xi}}	
$$
в точности равна 0. Так происходит потому, что функция под знаком интеграла нечетная, а пределы интегрирования - симметричны. В итоге остается
$$
\Phi(x_{0},y_{0})	=\cos\theta R\rho_{0}\int_{-\pi}^{\pi}d\xi\frac{\cos\xi}{\sqrt{R^{2}+r_{0}^{2}-2Rr_{0}\cos\xi}}.
$$
Точный ответ на интеграл в элементарных функциях не выражается, поэтому попробуем проанализировать его приближенно. Здесь то нам и понадобится условие $r_{0}\gg R$. Заметим, что слагаемое $2Rr_{0}\cos\xi$ под корнем мало в сравнении с $R^{2}+r_{0}^{2}$ по параметру $R/r_{0}\ll 1$. Поэтому, по нему можно раскладывать:
$$
\frac{1}{\sqrt{R^{2}+r_{0}^{2}-2Rr_{0}\cos\xi}}	=\frac{1}{\sqrt{R^{2}+r_{0}^{2}}}\frac{1}{\sqrt{1-\frac{2Rr_{0}\cos\xi}{R^{2}+r_{0}^{2}}}}\simeq\frac{1}{\sqrt{R^{2}+r_{0}^{2}}}+\frac{Rr_{0}\cos\xi}{\left(R^{2}+r_{0}^{2}\right)^{3/2}}+...$$
Заменим в пределе $r_{0}\gg R$ корень $\sqrt{R^{2}+r_{0}^{2}}\simeq r_{0}$. Тогда для потенциала получим
$$
\Phi(x_{0},y_{0})	=\cos\theta R\rho_{0}\int_{-\pi}^{\pi}d\xi\cos\xi\left(\frac{1}{r_{0}}+\frac{R}{r_{0}^{2}}\cos\xi+...\right).
$$
Первое слагаемое в скобке дает 0, потому что $\int_{-\pi}^{\pi}d\xi\cos\xi=0$. Это  отвечает тому, что полный заярд колечка равен 0. Второе слагаемое имеет вид
$$
\cos\theta R\rho_{0}\int_{-\pi}^{\pi}d\xi\cos^{2}\xi\frac{R}{r_{0}^{2}}	=\frac{\pi R^{2}\rho_{0}}{r_{0}^{2}}\cos\theta.
$$
Это конечное приближенное выражение для потенциала кольца.
Обратим внимание, что потенциал в этом случае явно зависит от направления, в котором находится “точка наблюдения”. Кроме того, потенциал спадает с расстоянием как $1/r_{0}^{2}$, в то время как потеницал точечного заряда ведет себя как $1/r_{0}$. Эта особенность связана с тем, что полный заряд колечка равен $0$. Давайте перепишем полученный ответ в векторной форме. Для этого введем вектор дипольного момента $\vec{d}$, который определим следующим образом
$$\vec{d}	=\left(\begin{array}{c}
\pi R^{2}\rho_{0}\\
0
\end{array}\right)$$
Тогда полный ответ для потенциала в пределе больших расстояний можно записать как
$$\Phi(\vec{r}_{0})	=\frac{\left(\vec{d}\cdot\vec{r}_{0}\right)}{r_{0}^{3}}$$
Это принятая запись для потенциала диполя. Отметим, что полученное выражение работает во всем пространстве, а не только в плоскоти $xy$.

\subsection*{Задача 5}

Пусть $C$ - замкнутая кривая, соответствующая обходу единичной окружности
против часовой стрелки. Найдём значение интеграла второго рода
\[
I=\ointop_{C}\frac{xdy-ydx}{x^{2}+y^{2}}
\]



\subsubsection*{Решение}


\paragraph{Неправильный способ}

Функция под интегралом является полным дифференциалом. Действительно:
\[
d\left(\arctan\frac{y}{x}\right)=\frac{1}{1+\left(\frac{y}{x}\right)^{2}}\cdot d\left(\frac{y}{x}\right)=\frac{1}{1+\left(\frac{y}{x}\right)^{2}}\cdot\frac{x\cdot dy-y\cdot dx}{x^{2}}=\frac{xdy-ydx}{x^{2}+y^{2}}
\]

 \noindent
поэтому, казалось бы, пользуясь формулой Ньютона-Лейбница, достаточно
взять разницу первообразной на концах контура $C$. Поскольку контур
замкнут, то эта разница тождественно равна нулю. Однако такой способ
рассуждений ошибочен, что связано с неаналитичностью ``первообразной''
в точках $x=0$ (и, соответственно, неаналитичностью подынтегральной
функции в начале координат). Этот интеграл играет важнейшую роль в
теории вычетов (теория функций комплексного переменного).


\paragraph{Правильный способ}

Поступим по определению. Запараметризуем контур $C$ как $C=\left\{ \mathbf{r}(\varphi),\varphi\in\left[0;2\pi\right]\right\} $
и $\mathbf{r}(\varphi)=(\cos\varphi,\sin\varphi)$. В таком случае
интеграл запишется как:
\[
I=\int_{0}^{2\pi}\frac{\cos\varphi\cos\varphi d\varphi+\sin\varphi\sin\varphi d\varphi}{\sin^{2}\varphi+\cos^{2}\varphi}=\int_{0}^{2\pi}d\varphi=2\pi\neq0
\]

\subsection*{Гауссов интеграл в многомерном случае}
Рассмотрим более общий случай многомерного Гауссова интеграла

\begin{equation}
I=\int dx_{1}...dx_{n}e^{-x^{T}Ax}
\end{equation}
Интегрирование здесь ведется в бесконечных пределах по всем осям, а
\[
x=\left(\begin{array}{c}
x_{1}\\
...\\
x_{n}
\end{array}\right)
\]
Матрица $A$ -- квадратная и при этом симметричная. Чтобы такой интеграл сходился матрица должна быть положительно определенной (иначе есть направление вдоль которого подынтегральная функция либо не меняется (в лучшем случае), либо вообще растет экспоненциально. По теореме из линейной алгебры для любой симметричной матрицы (в комплексном -- для случае эрмитовой) есть ортогональное преобразование координат, диагонализующее её. Т.е.

\begin{equation}
A_\lambda=S^{-1}AS=\mathrm{diag}\{\lambda_{1},...,\lambda_{n}\}\text{,\quad\quad}S^{-1}=S^{T}
\end{equation}
Раз $A$ положительно определенная, то
$\lambda_{i}>0$. Заметим, что
\[
x^{T}Ax=x^{T}SS^{-1}ASS^{-1}x=(S^{-1}x)^{T}S^{-1}AS(S^{-1}x)=(S^{-1}x)^{T}A_\lambda(S^{-1}x)
\]
Назовем $y=S^{-1}x$. Это определяет ортогональную замену координат,
$J=|\mathrm{det}(S)|=1$. Тогда

\begin{equation}
I=\int dy_{1}...dy_{n}e^{-y^{T}A_\lambda y}
\end{equation}
Ввиду диагональности
$A_\lambda$,

\begin{equation}
I=\int dy_{1}e^{-\lambda_{1}y_{1}^{2}}...\int dy_{n}e^{-\lambda_{n}y_{n}^{2}}=\sqrt{\frac{\pi^{n}}{\lambda_{1}...\lambda_{n}}}
\end{equation}
Известно, что
$\mathrm{det}\:A=\lambda_{1}\lambda_{2}...\lambda_{n}$ (дейсвтительно, детерминант не зависит от замены базиса, а в диагональном виде он вычисляется легко). В итоге, окончательно получаем

\begin{equation}
I=\sqrt{\frac{\pi^{n}}{\mathrm{det}\: A}}
\end{equation}
Такой подход к Гауссову интегралу без труда позволяет обобщить метода перевала на случай функций с несколькими переменными. Пусть $f(x_{1}...x_{n})$ имеет один максимум в точке $x_{0}$. Требуется вычислить

\begin{equation}
I=\int dx_{1}dx_{2}...dx_{n}e^{Nf(x)}
\end{equation}
при $N\rightarrow\infty$. Считаем, что точка $x_{0}$
лежит достаточно глубоко в области интегрирования так, что пределы можно распространить до бесконечности. Идея, как и раньше, состоит в том, чтобы разложить функцию $f$ в окрестности максимума в ряд Тейлора и вычислить Гауссов интеграл. Законность процедуры обеспечивается условием $N\rightarrow\infty$. Вычисляем:

\begin{equation}
f(x)=f(x_{0})-(x-x_{0})^{T}A(x-x_{0})
\end{equation}
где

\begin{equation}
A=-\frac{1}{2}\left(\begin{array}{ccc}
\frac{\partial^{2}f}{\partial x_{1}^{2}}(x_{0}) & ... & \frac{\partial^{2}f}{\partial x_{1}\partial x_{n}}(x_{0})\\
... & ... & ...\\
\frac{\partial^{2}f}{\partial x_{n}\partial x_{1}}(x_{0}) & ... & \frac{\partial^{2}f}{\partial x_{n}^{2}}(x_{0})
\end{array}\right)
\end{equation}
Матрица $A$ - симметричная (независимость смешанных производных от порядка дифференцирования) и положительно определенная ($x_0$ - максимум функции $f$). Тогда
\begin{equation}
I\approx\sqrt{\frac{\pi^{n}}{N^{n}\mathrm{det}\: A}}e^{Nf(x_{0})}
\end{equation}
Чем больше $N$, тем лучше работает формула. Если перевальных точек несколько, то по ним надо просуммировать.


\subsection*{Задачи для домашнего решения}

\noindent \textbf{Упражнение 1}

\noindent Вычилслите якобиан перехода к сферическим координатам

\begin{equation}\notag
x=r\cos\phi\sin\theta,\quad y=r\sin\phi\sin\theta,\quad z=r\cos\theta.
\end{equation}

\noindent Вычилслите якобиан перехода к тороидальным координатам

\begin{equation}\notag
x=(R+\rho\sin\theta)\cos\phi,\quad y=(R+\rho\sin\theta)\sin\phi,\quad z=\rho\cos\theta.	
\end{equation}

\noindent Здесь $R$ - фиксированный параметр. Такие координаты не являются однозначными, однако все равно бывают полезны (см. задачу 1).

\vspace{15pt}
\noindent \textbf{Упражнение 2}

\noindent Вычислите интеграл по шару, зависящий от “волнового вектора” $\vec{k}$:
\begin{equation}\notag
I(\vec{k})=\int_{|\vec{r}|<r_{0}}d^{3}re^{i\vec{k}\vec{r}}	
\end{equation}

\noindent \textit{Примечание: такой интеграл является примером преобразования Фурье в 3D:}
\begin{equation}\notag
I(\vec{k})	=\mathcal{F}_{\vec{k}}\{\theta(r_{0}-|\vec{r}|)\},\quad\theta(x)=\begin{cases}
\begin{array}{c}
1,\quad x\geq0\\
0,\quad x<0
\end{array}\end{cases}
\end{equation}
\noindent \textit{О преобразовании Фурье мы поговорим на одном из следующих семинаров.}

\noindent Более того этот интеграл встречается в простейшей задаче квантовомеханического рассеяния.

\vspace{15pt}
\noindent \textbf{Упражнение 3}

\noindent Вычислите длину спирали:

\begin{equation}
x=\cos t\text{,}\quad y=\sin t,\quad z=t
\notag
\end{equation}	

\noindent где $t\in[0,2\pi]$. Вычислите массу этой спирали, если ее плотность $\rho(t)=\cos^{2}t$.

\noindent Вычислите по ней криволинейный интеграл второго рода от функции
\begin{equation} \notag
\vec{F}(\vec{r})	=\left(\begin{array}{c}
-yz/(x^{2}+y^{2})\\
xz/(x^{2}+y^{2})\\
z(1+x^{2}-y^{2})
\end{array}\right).
\end{equation}

\vspace{15pt}
\noindent \textbf{Задача 1}

\noindent a) Вычислите момент инерции тора с однородной объемной массовой плотностью $\rho$ относительно оси его симметрии $z$. Тор задается следующими уравнениями:

\begin{equation}\notag
x	=(R+\rho\sin\theta)\cos\phi,\quad y=(R+\rho\sin\theta)\sin\phi,\quad z=\rho\cos\theta,
\end{equation}
\noindent где $\rho<r_{0}<R$, $\theta\in[0,2\pi]$, $\phi\in[0,2\pi]$.

\noindent Эта задача решена в материалах семинара, но Вам было бы поучительно с ней разобраться самим.

\noindent b) Вычислите момент инерции тора относительно осей $x$ и $y$.

\noindent \textit{Подсказка: удобно воспользоваться тождеством} $I_{x}+I_{y}+I_{z}=2I_{0}$.

\vspace{15pt}
\noindent \textbf{Задача 2}

\noindent Вычислите потенциал, создаваемый неоднородно заряженной сферой радиуса $r_{0}$ с поверхностной плотностью заряда (заданной в сферических координатах)

\begin{equation}\notag
\sigma(\vec{r})|_{|\vec{r}|=r_{0}}	=\sigma_{0}\cos\theta
\end{equation}
\noindent в точке $\vec{R}$ с координатами
\begin{equation}
\begin{cases}
X=R\sin\psi,\\
Y=0,\\
Z=R\cos\psi,
\end{cases}\notag
\end{equation}
\noindent где $R>r_{0}$, а $\psi$ - заданный угол. 

\noindent В данном случае потенциал можно посчитать по формуле:

\begin{equation}\notag
\phi(\vec{R})	=k\int_{|\vec{r}|=r_{0}}\frac{\sigma(\vec{r})dS}{|\vec{R}-\vec{r}|}\text{.}
\end{equation}

\noindent Здесь $dS$ - бесконечно малый элемент площади сферы: 
\begin{equation}
\notag
dS	=r_{0}^{2}\sin\theta d\theta d\phi\text{.}
\end{equation}
\noindent Обратите внимание: $r_{0}^{2}\sin\theta$ - это якобиан перехода к сферическим координатам вычисленный при $r=r_{0}$.

\noindent Задачу можно в пару строчек решить** без явного вычисления интеграла. Единственное, что надо будет сделать - посчитать несложную производную.

\vspace{15pt}
\noindent \textbf{Задача 3}

\noindent Вычислите интеграл по квадрату $x\in[0,2\pi]$, $y\in[0,2\pi]$:

\begin{equation}\notag
I	=\int\frac{dx}{2\pi}\frac{dy}{2\pi}\frac{1-\cos(x+y)}{2-\cos x-\cos y}
\end{equation}
\noindent \textit{Подсказка: для решения может быть удобно перейти к повернутой системе координат}
\begin{equation}\notag
u	=\frac{x+y}{2},\quad v=x-y.
\end{equation}
\end{document}
